%%%%%%%%%%%%%%%%%%%%%%%%%%%%%%%%%%%%%%%%%
% Journal Article
% LaTeX Template
% Version 1.3 (9/9/13)
%
% This template has been downloaded from:
% http://www.LaTeXTemplates.com
%
% Original author:
% Frits Wenneker (http://www.howtotex.com)
%
% License:
% CC BY-NC-SA 3.0 (http://creativecommons.org/licenses/by-nc-sa/3.0/)
%
%%%%%%%%%%%%%%%%%%%%%%%%%%%%%%%%%%%%%%%%%

%----------------------------------------------------------------------------------------
%	PACKAGES AND OTHER DOCUMENT CONFIGURATIONS
%----------------------------------------------------------------------------------------

\documentclass[twoside]{article}

\usepackage{lipsum} % Package to generate dummy text throughout this template

\usepackage[sc]{mathpazo} % Use the Palatino font
\usepackage[T1]{fontenc} % Use 8-bit encoding that has 256 glyphs
\linespread{1.05} % Line spacing - Palatino needs more space between lines
\usepackage{microtype} % Slightly tweak font spacing for aesthetics

\usepackage[hmarginratio=1:1,top=32mm,columnsep=20pt]{geometry} % Document margins
\usepackage{multicol} % Used for the two-column layout of the document
\usepackage[hang, small,labelfont=bf,up,textfont=it,up]{caption} % Custom captions under/above floats in tables or figures
\usepackage{booktabs} % Horizontal rules in tables
\usepackage{float} % Required for tables and figures in the multi-column environment - they need to be placed in specific locations with the [H] (e.g. \begin{table}[H])
\usepackage{hyperref} % For hyperlinks in the PDF

\usepackage{lettrine} % The lettrine is the first enlarged letter at the beginning of the text
\usepackage{paralist} % Used for the compactitem environment which makes bullet points with less space between them

\usepackage{abstract} % Allows abstract customization
\renewcommand{\abstractnamefont}{\normalfont\bfseries} % Set the "Abstract" text to bold
\renewcommand{\abstracttextfont}{\normalfont\small\itshape} % Set the abstract itself to small italic text

\usepackage{titlesec} % Allows customization of titles
\renewcommand\thesection{\Roman{section}} % Roman numerals for the sections
\renewcommand\thesubsection{\Roman{subsection}} % Roman numerals for subsections
\titleformat{\section}[block]{\large\scshape\centering}{\thesection.}{1em}{} % Change the look of the section titles
\titleformat{\subsection}[block]{\large}{\thesubsection.}{1em}{} % Change the look of the section titles

\usepackage{fancyhdr} % Headers and footers
\pagestyle{fancy} % All pages have headers and footers
\fancyhead{} % Blank out the default header
\fancyfoot{} % Blank out the default footer
\fancyhead[C]{} % Custom header text
\fancyfoot[RO,LE]{\thepage} % Custom footer text

%----------------------------------------------------------------------------------------
%	TITLE SECTION
%----------------------------------------------------------------------------------------

\title{\vspace{-15mm}\fontsize{24pt}{10pt}\selectfont\textbf{Motivations and Design behind Agar}} % Article title

\author{
\large
\textsc{Alexander Schneider}\thanks{All praise RNGeesus}\\[2mm] % Your name
\normalsize Heinrich Heine University D\"usseldorf \\ % Your institution
\normalsize \href{mailto:alexander.schneider@hhu.de}{alexander.schneider@hhu.de} % Your email address
\vspace{-5mm}
}
\date{}

%----------------------------------------------------------------------------------------

\begin{document}

\maketitle % Insert title

\thispagestyle{fancy} % All pages have headers and footers

%----------------------------------------------------------------------------------------
%	ABSTRACT
%----------------------------------------------------------------------------------------

\begin{abstract}

\noindent This paper details the reasons and motivations behind the petri-net Simulator Agar.\\
Agar is capable of creating and simulating place/transition petri nets. Because Agar is written
in Clojure one of the main design paradigms was to keep Agar \textbf{Simple}.  

\end{abstract}

%----------------------------------------------------------------------------------------
%	ARTICLE CONTENTS
%----------------------------------------------------------------------------------------

\begin{multicols}{2} % Two-column layout throughout the main article text

\section{Introduction}

\lettrine[nindent=0em,lines=3]{C} lojure is a programming language which has simplicity as one of its main paradigms.
Following this tradition we introduce Agar, a simple place/transition petri net simulator written in Clojure.
Simple programming tries to minimize the interactions of a method. An implication of this is to minimize
state, because state causes unknown and uncontrollable interactions.
With simplicity in mind Agar is divided into several Clojure files which fulfill a different role described in detail 
in the following chapters. 

%------------------------------------------------

\section{Data Structure}
The Data structure which is used to represent and save the petri-nets was developed, before the programming and
the design of Agar itself began. Only the property part changed during implementation and is described further
in the Refactoring chapter.\\
The entirety of all nets are represented as a hash-map. The hash-map contains the netnames as keys and the nets as values.
A single net is a hash-map as well with the keys :places, :transitions, :edges-in, :edges-out, and :props.
The value of :places is in turn a hash-map which holds the name of the place as a key and the number of tokens currently
in the place as a value.
The value of :transition is a set, that contains all transition-names. 
The values of :edges-in and :edges-out are also hash-maps. The keys of those hash maps are the transitions where the edges are going in or
out of respective. The values there are in turn (you probably guessed it right) hash-maps with the places where
the edge is coming from or going in respectively as keys and the cost of the edge as values.
The value of :props is a set containing the property-strings.
An Example data-structure looks like this:
\begin{verbatim}
{"foo" {:places {"baz" 5, "bar" 5}, 
		:transitions #{"bam"}, 
		:edges-in {"bam" {"baz" 1, "bar" 1}}, 
		:edges-out {"bam" {"baz" 1}}, 
		:props #{"(notalive "foo")"}}
 "test2" {:places {"fizz" 9, buzz "10"},
 		  :transitions #{"fizzbuzz"},
 		  :edges-in {}
 		  :edges-out {}
 		  :props #{}}}
\end{verbatim}

%------------------------------------------------

\section{Functionality}
Agar can be used via REPL using the simulator.clj or via GUI using the gui.clj.
Both librarys have the almost the same functionality.
It is possible to create nets, add places, transitions and edges and properties. Nets can also be 
deleted, copied, and merged. 
Additionally user-selected or random transitions can be fired.\\ 
The GUI also auto-evaluates all properties after a transition has been fired. This function was omitted in
the simulator, because in the REPL thats just one call to \texttt{(eval (read-string))}.
Firing of a random transition x times is also omitted in the Simulator. Again because it is just a simple
call of \texttt{(times)} on the random-fire function.\\
Every Net can be saved or loaded separately in addition to saving or loading the whole database of nets.

\section{Implementation}

Agar is divided mainly into 4 parts. 
\begin{compactitem}
\item Core
\item API
\item Simulator 
\item GUI
\end{compactitem}
\subsection{Core module}
The \texttt{core.clj} is the heart of the application. Its sole purpose is the manipulation of the
domain specific language. For example creating new petri nets, adding and deleting nodes and edges,
saving and loading nets, etc.
The core does not check whether the constructed petri nets are valid or not, it simply executes the Functions.
Because of this every function either just has to execute one simple set of instructions or is a composition of
aforementioned methods.
Furthermore the core manages a database of nets inside an atom. 
The \texttt{merge-nets} function is the only one which could be considered not simple. But since it only
acts as a means to "bundle" other functions it should be acceptable nonetheless.

\subsection{API}
The \texttt{api.clj} encapsulates the core. The main function of the API is the filtering of the
input to prevent faulty petri nets. Every function that involves user input checks whether this 
input is valid (e.g. adding  place with a name already present in the net is not valid). When the input 
is valid the function just passes the call to the corresponding core function.

\subsection{Simulator}
The \texttt{simulator.clj} is one possible application/use case of the API.
It takes the abstract of a petri net that the API provides and uses it for the
concrete case of a simulator. \\
The simulator provides properties that can be added to the petri net. Furthermore
the ability to "fire" a transition is provided. The simulator uses the API functions to
change the state of the net to simulate the execution of a transition.

\subsection{GUI}
The \texttt{gui.clj} provides a graphical interface for the simulator. This is the point
where the simplicity breaks. This file heavily utilizes the \texttt{seesaw} library. Seesaw
is a clojure library that builds on the Java Swing Toolkit. Swing does not comply with simplicity
guidelines. While using swing there is no way to separate the code for the appearance and the code which fulfills 
a function. \\
While seesaw helps make swing simpler it can not change the core design of swing. In the case of Agar we tried
to first design the looks of the GUI and then attach the functions trough listeners in the main Function.
This results in the main function being the only point where function and appearance are intertwined. 

\section{Refactorings}
Due to previous planing described in Chapter II there was only one big refactoring in the design of Agar.
The handling of the properties changed during the implementation of the GUI. Before that the properties were
saved as a sequence which evaluates to a function call. Because of the nature of the GUI and the decision to give the
user the ability to operate on all possible properties with \texttt{or} and \texttt{not} through  mouse-clicks the
representation changed. The new representation was a String representing the sequence that evaluates to a call
executing the property functions. Because of that all property handling functions in the core had to be adapted to the new format.
%------------------------------------------------


%----------------------------------------------------------------------------------------

\end{multicols}

\end{document}
